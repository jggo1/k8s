\chapter{KUBERNETES}
\label{chap:kubernetes}

O Kubernetes é uma plataforma portátil, extensível e de código aberto para gerenciar cargas de trabalho e serviços em contêiner, que facilita a configuração declarativa e a automação. Possui um ecossistema grande e de rápido crescimento. Os serviços, suporte e ferramentas do Kubernetes estão amplamente disponíveis.

O nome Kubernetes é originário do grego, significando timoneiro ou piloto. O Google abriu o projeto Kubernetes em 2014. O Kubernetes tem uma década e meia de experiência que o Google tem em executar cargas de trabalho de produção em grande escala, combinadas com as melhores idéias e práticas da comunidade.

Ao implantar o Kubernetes, você obtém um cluster.

Um cluster é um conjunto de máquinas, chamadas nós, que executam aplicativos em contêiner gerenciados pelo Kubernetes. Um cluster possui pelo menos um nó de trabalho e pelo menos um nó principal.



Este documento descreve os vários componentes necessários para ter um cluster Kubernetes completo e funcionando.

Aqui está o diagrama de um cluster Kubernetes com todos os componentes unidos.

\section{COMPONENTES PRINCIPAIS}

Os componentes principais fornecem o plano de controle do cluster. Os componentes principais tomam decisões globais sobre o cluster (por exemplo, agendamento) e detectam e respondem a eventos do cluster (por exemplo, iniciando um novo pod quando o campo de réplicas de uma implantação não está satisfeito).

Os componentes principais podem ser executados em qualquer máquina no cluster. No entanto, para simplificar, os scripts de configuração geralmente iniciam todos os componentes principais na mesma máquina e não executam contêineres de usuários nessa máquina. Consulte Criando clusters de alta disponibilidade para obter um exemplo de instalação da VM com várias master.

\subsection{kube-apiserver}
O servidor da API é um componente do plano de controle do Kubernetes que expõe a API do Kubernetes. O servidor da API é o front end do plano de controle do Kubernetes.

A principal implementação de um servidor de API do Kubernetes é o kube-apiserver. O kube-apiserver foi desenvolvido para ser dimensionado horizontalmente, ou seja, é dimensionado com a implantação de mais instâncias. Você pode executar várias instâncias do kube-apiserver e equilibrar o tráfego entre essas instâncias.

\subsection{etcd}

Armazenamento de valores-chave consistente e de alta disponibilidade usado como repositório do Kubernetes para todos os dados do cluster.

Se o cluster do Kubernetes usa o etcd como repositório de backup, verifique se você tem um plano de backup para esses dados.

Você pode encontrar informações detalhadas sobre o etcd na documentação oficial

\subsection{kube-scheduler}

Componente no mestre que assiste os pods recém-criados que não têm nenhum nó designado e seleciona um nó para execução.

Os fatores levados em consideração nas decisões de agendamento incluem requisitos de recursos individuais e coletivos, restrições de hardware/software/política, especificações de afinidade e anti-afinidade, localidade dos dados, interferência entre cargas de trabalho e prazos.

\subsection{kube-controller-manager}

Componente no mestre que executa controladores.

Logicamente, cada controlador é um processo separado, mas para reduzir a complexidade, todos são compilados em um único binário e executados em um único processo.

Esses controladores incluem:
\begin{itemize}
	\item Controlador de nó: responsável por perceber e responder quando os nós são desativados.
	\item Controlador de replicação: responsável por manter o número correto de pods para cada objeto do controlador de replicação no sistema.
	\item Controlador de pontos finais: preenche o objeto Pontos finais (ou seja, junta-se a Serviços e pods).
	\item Conta de serviço e controladores de token: crie contas padrão e tokens de acesso à API para novos namespaces
\end{itemize}

\subsection{cloud-controller-manager}

O cloud-controller-manager executa controladores que interagem com os provedores de nuvem subjacentes. O binário cloud-controller-manager é um recurso alfa introduzido no Kubernetes versão 1.6.

O cloud-controller-manager executa apenas loops de controlador específicos do provedor de nuvem. Você deve desativar esses loops de controlador no kube-controller-manager. Você pode desativar os loops do controlador configurando o sinalizador --cloud-provider como externo ao iniciar o kube-controller-manager.

o gerenciador de controlador de nuvem permite que o código do fornecedor de nuvem e o código Kubernetes evoluam independentemente um do outro. Em versões anteriores, o código principal do Kubernetes dependia do código específico do provedor de nuvem para funcionalidade. Em versões futuras, o código específico para os fornecedores de nuvem deve ser mantido pelo próprio fornecedor de nuvem e vinculado ao gerenciador de controlador de nuvem durante a execução do Kubernetes.

Os seguintes controladores têm dependências do provedor de nuvem:

\begin{itemize}
	\item Node Controller: For checking the cloud provider to determine if a node has been deleted in the cloud after it stops responding
	\item Route Controller: For setting up routes in the underlying cloud infrastructure
	\item Service Controller: For creating, updating and deleting cloud provider load balancers
	\item Volume Controller: For creating, attaching, and mounting volumes, and interacting with the cloud provider to orchestrate volumes
\end{itemize}

\section{Componentes do nó}

Node components run on every node, maintaining running pods and providing the Kubernetes runtime environment.

\subsection{kubelet}
Um agente que é executado em cada nó no cluster. Ele garante que os contêineres estejam sendo executados em um pod.

O kubelet utiliza um conjunto de PodSpecs que são fornecidos por vários mecanismos e garante que os contêineres descritos nesses PodSpecs estejam funcionando e funcionando corretamente. O kubelet não gerencia contêineres que não foram criados pelo Kubernetes.

\subsection{kube-proxy}

O kube-proxy é um proxy de rede que é executado em cada nó do cluster, implementando parte do conceito de Serviço Kubernetes.

O kube-proxy mantém regras de rede nos nós. Essas regras de rede permitem a comunicação em rede com seus Pods a partir de sessões de rede dentro ou fora do cluster.

O kube-proxy usa a camada de filtragem de pacotes do sistema operacional, se houver uma disponível. Caso contrário, o kube-proxy encaminha o próprio tráfego.

\subsection{Container Runtime}

O tempo de execução do contêiner é o software responsável pela execução de contêineres.

O Kubernetes suporta vários tempos de execução do contêiner: Docker, container, cri-o, rktlet e qualquer implementação do Kubernetes CRI (Container Runtime Interface).