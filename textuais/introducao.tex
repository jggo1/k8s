\chapter{INTRODUÇÃO}
\label{chap:introducao}

Edite e coloque aqui o seu texto de introdução.

A Introdução é a parte inicial do texto, na qual devem constar o tema e a delimitação do assunto tratado, objetivos da pesquisa e outros elementos necessários para situar o tema do trabalho, tais como: justificativa, procedimentos metodológicos (classificação inicial), embasamento teórico (principais bases sintetizadas) e estrutura do trabalho, tratados de forma sucinta. Recursos utilizados e cronograma são incluídos quando necessário. Salienta-se que os procedimentos metodológicos e o embasamento teórico são tratados, posteriormente, em capítulos próprios e com a profundidade necessária ao trabalho de pesquisa.

\section{JUSTIFICATIVA}
\label{sec:justificativa}

\section{OBJETIVO}
\label{sec:objetivo}

\section{PROBLEMÁTICA}
\label{sec:problematica}

\section{METODOLOGIA}
\label{sec:metodologia}

\section{ORGANIZAÇÃO DO TRABALHO}
\label{sec:organizacao}

Normalmente ao final da introdução é apresentada, em um ou dois parágrafos curtos, a organização do restante do trabalho acadêmico.
Deve-se dizer o quê será apresentado em cada um dos demais capítulos.
